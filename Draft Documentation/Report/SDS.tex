%%%%%%%%%%%%%%%%%%%%%%%%%%%%%%%%%%%%%%%%%%%%%%%%%%%%%%%%%%%%%%%%%%%%%%%%%%%%%%%
%
% witseiepaper-2005.tex
%
%                       Ken Nixon (12 October 2005)
%
%                       Sample Paper for ELEN417/455 2005
%
%%%%%%%%%%%%%%%%%%%%%%%%%%%%%%%%%%%%%%%%%%%%%%%%%%%%%%%%%%%%%%%%%%%%%%%%%%%%%%%%

\documentclass[10pt, onecolumn]{witseiepaper}

%
% All KJN's macros and goodies (some shameless borrowing from SPL)
\usepackage{KJN}


%%%%%%%%%%%%%%%%%%%%%%%%%%%%%%%%%%%%%%%%%%%%%%%%%%%%%%%%%%%%%%%%%%%%%%%%%%%%%%%
\begin{document}


\title{FLEET OF TAXI SERVICE MANAGEMENT SYSTEM - SOFTWARE DESIGN SPECIFICATION}

\author{Danielle Winter (563795), Frederick Nieuwoudt (386372), Stephen Friedman (360938) \& Sello Molele (0604606X)}
\thanks{School of Electrical \& Information Engineering, University of the
Witwatersrand, Private Bag 3, 2050, Johannesburg, South Africa}


%%%%%%%%%%%%%%%%%%%%%%%%%%%%%%%%%%%%%%%%%%%%%%%%%%%%%%%%%%%%%%%%%%%%%%%%%%%%%%%
%

\maketitle
\thispagestyle{empty}\pagestyle{empty}


%%%%%%%%%%%%%%%%%%%%%%%%%%%%%%%%%%%%%%%%%%%%%%%%%%%%%%%%%%%%%%%%%%%%%%%%%%%%%%%
%
\section{INTRODUCTION}
\subsection{Purpose of this Document}
This Software Design Specification (SDS) details the design for the WitsCABS application. The back- and front-end interfaces are described separately. The front-end and back-end integration is also described. The remainder of Section 1 details the project scope and abbreviations and definitions used in the remainder of the SDS. Section 2 addresses the basic system architecture and Section 3 gives a more detailed description of how the system will be implemented. 
\subsection{Project Scope}
The taxi fleet management system is a tool that provides a service where a user can request a taxi and the system will dispatch the closest available driver.\\\\
The system back-end will be required to store information about drivers, customers, and the current pricing scheme. The system will keep historical data regarding trips that have been taken for auditing and dispute resolution.\\\\
The system front end will be presented in the form of a web interface. This will provide a means for Dispatch Control Centre agents to input new customer's details and for drivers to manage their current job requests. The web interface will also provide a quoted price for the client.\\\\
For the purposes of this project, the entire system functionality is described in the SDS. However, only a few modules and components of the code are implemented as a prototype of the system in order to show basic functionality.

\subsection{Definitions}
\subsubsection{Definitions}
\begin{itemize}
\item WitsCABS: The name of the software application including front- and back-ends
\item Driver: The taxi-cab drivers employed by the WitsCABS company to pick-up and deliver customers
\item Customer: Clients of the company who request lifts
\end{itemize}
\subsubsection{Abbreviations}
\begin{itemize}
\item DCC: Dispatch Control Centre
\end{itemize}


\section{SYSTEM ARCHITECTURE}
\subsection{Front-end Views}
The front-end of the WitsCABS application is presented as two separate views in an HTML web-application. The front end is coded in angular.js using Twitter Bootstrap for a unified visual appearance and pop-up functionality. The respective interfaces are the DCC view and the Driver view. These views are linked through a common log-in screen view where the respective users select and log in to their interface.
\subsubsection{DCC View}
The DCC view includes panels for a new customer form and an active job list. The "new customer form" panel has text boxes for the customer's name, current location, destination and cellphone number. A submit button allows the DCC agent to submit the entry to the back-end database, at which point, the distance and price for the job is calculated. The new job will be added to the "active jobs" panel in the DCC interface and the DCC will be able to see the price calculated for the customer.
\subsubsection{Driver View}
The driver view allows the driver to see their assigned job, the customer name and contact details and their travel destination. In addition to this, two buttons are presented to the driver - one for "On the Way" when they leave the taxi rank and one for "Job Complete" when the customer has been delivered.
\subsubsection{Log-in View}
The log-in page gives the user the ability to select whether they are a driver or a DCC agent and allows them to log in to their respective view. The log-in page should prompt users for a username and password and their entry should be authenticated through the back-end.

\subsection{Back-end Implementation}
The back-end of the WitsCABS application consists of two main parts. The part that directly interfaces with the front-end is an Apache web server which runs on Ubuntu and executes Python scripts in order to interact with the database and external APIs. The part of the back-end that stores the data is an SQLite database.

\subsubsection{Web Server}\mbox{}\\
The role of the web server in the WitsCABS application is primarily to act as an intermediary between the front-end and the database. It is also required to connect to external APIs in order to deliver functionality that would be too complex to implement in the application.\\ \\
The three external APIs are Paym8, SMSPortal and Google Maps. The Paym8 API is used to charge customers for Trips that are taken using WitsCABS. The SMSPortal API is used to notify Customers about the status of their Driver. The Google Maps API is used to translate addresses in to GPS Coordinates, as well as route from one location to another and determine the real distance between two points.

\subsubsection{Database}\mbox{}\\
The role of the database in the WitsCABS application is to store data so that different users can interact with the same data without directly transmitting the data between each other and so that data can be kept for auditing and dispute resolution if need be.
\\ \\
For the database queries, either many simple queries could be implemented or few complex queries could be implemented. Complex queries were chosen as they take advantage of optimisation in SQL. Complex queries require more development and testing time, but once implemented and verified produce faster results than many simple queries.

\section{DETAILED ARCHITECTURE AND FEATURES}
\subsection{Front-end Architecture}
The front-end architecture is detailed below for the angular.js and HTML files. The front-end and back-end communicate through button pushes - firstly with the log-in page user authentication and then in the respective views.
\subsubsection{angular.js File}
\begin{itemize}
\item angular module - WitsCABS
\item angular controller - JobController
\item customer directive - jobs
\item array of customers
\begin{itemize}
\item customer name (string)
\item customer destination (string)
\item customer current location (string)
\item customer phone number (string)
\item distance (integer)
\item price (integer)
\item job active (boolean)
\item driver assigned (boolean)
\end{itemize}
\end{itemize}
%%%%%%%%%%%%%%%%%%%%%%%%%%%%%%%%%%%%%%%%%%%%%%%%%%%%%%%%%%%%%%%%%%%%%%%%%%
\subsubsection{HTML File}\mbox{}\\

When DCC agents submit a new customer in the form, the back-end database is updated and a taxi-driver is assigned to the new job. The driver interface will be updated. Drivers can state when a job is complete, at which point, the database will once again be updated and the respective views refreshed. Upon log-in details being entered, authentication is implemented through the back-end framework before users can be redirected to their respective pages.
\begin{itemize}
\item DCC View
\begin{itemize}
\item Headings\\
\item Active job panel
\begin{itemize}
\item active customers list
\item driver assigned (boolean)
\item customer location, destination and phone number (paragraphs under each customer)\\
\end{itemize}
\item New customer form panel
\begin{itemize}
\item customer name field
\item customer destination field
\item customer current location field
\item customer phone number field
\item submit button \\
\end{itemize}
\end{itemize} 
\item Driver View
\begin{itemize}
\item Headings\\
\item Assigned job panel
\begin{itemize}
\item customer details
\item "On the way" button
\item "Job complete" button\\
\end{itemize}
\end{itemize}
\item Log-in Page
\begin{itemize}
\item Log-in modal
\begin{itemize}
\item driver log-in button - redirects to driver interface
\item DCC log-in button - redirects to DCC interface
\item close button\\
\end{itemize}
\end{itemize}
\end{itemize}


\subsection{Back-end Architecture}
\subsubsection{Web Server}\mbox{}\\
The Web Server functions primarily as an intermediary between the front-end and the database. It fulfils this role through a collection of interfaces that the front-end can access. The Interfaces are as follows:
\begin{itemize}
\item Login: This is to check that the User attempting to log in is a valid User and check which type of User it is, if successful.
	\begin{itemize}
	\item Inputs:
		\begin{itemize}
		\item UserName - A string containing the User's login name
		\item PasswordHash - A string containing the User's hashed password
		\end{itemize}
	\item Output:
		\begin{itemize}
		\item String containing either "Driver", "Agent", or "Rejected" 
		\end{itemize}
	\end{itemize}
\item GetActiveTrips: This is to retrieve a list of all Trips that are currently active.
	\begin{itemize}
	\item Inputs:
		\begin{itemize}
		\item None required
		\end{itemize}
	\item Output:
		\begin{itemize}
		\item An array containing Trip objects
		\end{itemize}
	\end{itemize}
	
\item CreateTrip: This is to store the information about a new trip when a DCC Agent captures it.
	\begin{itemize}
	\item Inputs:
		\begin{itemize}
		\item StartLocation - A string containing the friendly name of the location
		\item EndLocation - A string containing the friendly name of the location
		\item CustomerName - A string containing the Customer's name
		\item ContactNumber - A string containing the Customer's cell phone number
		\item CreditCardNumber - A string containing the Customer's credit card number
		\item CreditCardType - A string containing the Customer's credit card type ("Visa" or "Mastercard" expected)
		\item CreditCardCvv - A string containing the Customer's credit card CVV number
		\end{itemize}
	\item Output:
		\begin{itemize}
		\item A boolean indicating if the Trip was created successfully or not
		\end{itemize}
	\end{itemize}
\item GetDriverTrip: This is to retrieve the Trip that is assigned to a particular Driver.
	\begin{itemize}
	\item Inputs:
		\begin{itemize}
		\item DriverId - an integer indicating the system's internal id of the Driver
		\end{itemize}
	\item Output:
		\begin{itemize}
		\item A Trip object
		\end{itemize}
	\end{itemize}
\item SendNotification: This is to send an sms to a Customer.
	\begin{itemize}
	\item Inputs:
		\begin{itemize}
		\item TripId - an integer indicating the system's internal id of the Trip
		\end{itemize}
	\item Output:
		\begin{itemize}
		\item A boolean indicating if the Customer was notified successfully or not
		\end{itemize}
	\end{itemize}
\item SetTripComplete: This is to indicate that a Trip is completed and that a Customer can be charged and the Driver be made available to take other Trips.
	\begin{itemize}
	\item Inputs:
		\begin{itemize}
		\item TripId - an integer indicating the system's internal id of the Trip
		\end{itemize}
	\item Output:
		\begin{itemize}
		\item A boolean indicating if the Trip was successfully marked as completed or not
		\end{itemize}
	\end{itemize}
\end{itemize}

\subsubsection{Database}\mbox{}\\
The database consists of the following tables:
\begin{itemize}
\item Users: This table contains information to verify a User when they log in. The users' username and hashed password are stored. The hash of the password is stored so that nobody that has access to the WitsCABS database can compromise the users' passwords.
\item Agents: This table contains the real name of DCC agents and is used by the login process to determine if the user is a DCC agent.
\item Drivers: This table contains the real name of Drivers and is used by the login process to determine if the user is a Driver. This table also keeps track of the status and location of the Driver so that the best driver is chosen for each Trip.
\item Cost: This table contains rates that can apply to trips.
\item Customers: This table contains details for Customers, including contact information for SMS notification and financial information in order to charge Customers for Trips that are taken.
\item Locations: This table stores addresses against GPS coordinates, so that friendly names can used for informative purposes, while accurate GPS coordinates can be used for distance calculations.
\item Trips: This table contains information relating to each Trip undertaken. It allows DCC agents to view all active Trips, it allows Drivers to be easily assigned to Trips and it allows for the retention of historical data for auditing or dispute resolution purposes.
\end{itemize}

\subsubsection{Detailed Database Table Structure}
\subsection*{Users:}
\begin{tabular}{|c|l|c|}
\hline 
Field & Type & Notes\\ 
\hline 
UserId & Int & Primary Key, Auto-increment\\
\hline 
UserName & Text & Login Name of User\\
\hline 
PasswordHash & Text & Password, hashed to prevent unauthorised use\\
\hline
\end{tabular}

\subsection*{Agents:}
\begin{tabular}{|c|l|c|}
\hline 
Field & Type & Notes\\ 
\hline 
AgentId & Int & Primary Key, Auto-increment\\
\hline 
UserId & Int & Foreign Key\\
\hline 
Name & Text & Real name of DCC agent\\
\hline
\end{tabular}

\subsection*{Drivers:}
\begin{tabular}{|c|l|c|}
\hline 
Field & Type & Notes\\ 
\hline 
DriverId & Int & Primary Key, Auto-increment\\
\hline 
UserId & Int & Foreign Key\\
\hline 
Name & Text & Real name of Driver\\
\hline
CurrentLocationId & Int & Foreign Key\\
\hline
CurrentlyActive & Bit & Indicates if a driver is at work or not\\
\hline
CurrentlyAssigned & Bit & Indicates if driver is busy with a Trip\\
\hline
\end{tabular}

\subsection*{Cost:}
\begin{tabular}{|c|l|c|}
\hline 
Field & Type & Notes\\ 
\hline 
CostId & Int & Primary Key, Auto-increment\\
\hline 
Type & Text & Friendly description of Cost\\
\hline 
AmountInCents & Int & Money is indicated to 2 decimal places, so to avoid rounding issues,\\
• & • & work with money as an integer number of cents and divide by\\ 
• & • & 100 before displaying in order to show an amount in Rands\\
\hline
\end{tabular}

\subsection*{Customers:}
\begin{tabular}{|c|l|c|}
\hline 
Field & Type & Notes\\ 
\hline 
CustomerId & Int & Primary Key, Auto-increment\\
\hline 
Name & Text & Real name of Customer\\
\hline
PhoneNumber & Text & Contact number of Customer\\
\hline
CreditCardNumber & Text & Financial information of Customer\\
\hline
CreditCardType & Text & Financial information of Customer\\
\hline
CVV & Text & Financial information of Customer\\
\hline
\end{tabular}

\subsection*{Location:}
\begin{tabular}{|c|l|c|}
\hline 
Field & Type & Notes\\ 
\hline 
LocationId & Int & Primary Key, Auto-increment\\
\hline 
LocationName & Text & Friendly name of the Location\\
\hline
GPSLat & Int & GPS Latitude, stored in seconds\\
\hline
GPSLon & Int & GPS Longitude, stored in seconds\\
\hline
\end{tabular}

\subsection*{Trip:}
\begin{tabular}{|c|l|c|}
\hline 
Field & Type & Notes\\ 
\hline 
TripId & Int & Primary Key, Auto-increment\\
\hline 
CustomerId & Int & Foreign Key\\
\hline 
StartLocationId & Int & Foreign Key\\
\hline 
EndLocationId & Int & Foreign Key\\
\hline 
DriverId & Int & Foreign Key\\
\hline 
CostId & Int & Foreign Key\\
\hline 
Active & Bit & Indicates if the Trip is under-way or if it has been completed\\
\hline
\end{tabular}

%%%%%%%%%%%%%%%%%%%%%%%%%%%%%%%%%%%%%%%%%%%%%%%%%%%%%%%%%%%%%%%%%%%%%%%%%%%%%%%
%
%\nocite{*}
\bibliographystyle{witseie}
\bibliography{TaxiFleet}

%{\tiny \vfill \hfill \today \hspace{5mm} witseie-paper-2003.\TeX}

\end{document}

" vim: ts=4
" vim: tw=78
" vim: autoindent
" vim: shiftwidth=4
