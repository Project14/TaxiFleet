%%%%%%%%%%%%%%%%%%%%%%%%%%%%%%%%%%%%%%%%%
% title page.
%
%%%%%%%%%%%%%%%%%%%%%%%%%%%%%%%%%%%%%%%%%

%----------------------------------------------------------------------------------------
%	PACKAGES AND OTHER DOCUMENT CONFIGURATIONS
%----------------------------------------------------------------------------------------

\documentclass[12pt]{article}
\usepackage{graphicx} % Include images
\begin{document}

\begin{titlepage}

\newcommand{\HRule}{\rule{\linewidth}{0.5mm}} % Defines a new command for the horizontal lines, change thickness here

\center % Center everything on the page
 
%----------------------------------------------------------------------------------------
%	HEADING SECTIONS
%----------------------------------------------------------------------------------------
\begin{figure}
\centering
\includegraphics[scale=0.3]{wits-logo.jpg}
\end{figure}

\textsc{\LARGE University of the Witwatersrand}\\[0.5cm]
\textsc{\Large School of Electrical and Information Engineering}\\[0.5cm]
\textsc{\small Private Bag 3, 2050, Johannesburg, South Africa}\\[0.5cm]

%----------------------------------------------------------------------------------------
%	TITLE SECTION
%----------------------------------------------------------------------------------------

\HRule \\[0.5cm]
{ \huge \bfseries Taxi-cab Service Management System}\\ % Title of your document
\HRule \\[0.5cm]
 
%----------------------------------------------------------------------------------------
%	AUTHOR SECTION
%----------------------------------------------------------------------------------------

\begin{minipage}{0.4\textwidth}
\begin{flushleft} 
\emph{Front-End:}\\
Danielle Winter\\
Frederick Nieuwoudt\\[0.1cm]
\emph{Back-End:}\\
Stephen Friedman\\
Sello Molele
\end{flushleft}
\end{minipage}
~
\begin{minipage}{0.4\textwidth}
\begin{flushright} 
\emph{Student Number:} \\
563795\\
386372\\[0.6cm]
360938\\
0604606X
\end{flushright}
\end{minipage}\\[0.5cm]

{\large 11 April 2016}\\[0.5cm]
\newpage
\pagenumbering{gobble}
\begin{abstract}
A software application for a taxi-cab company called WitsCABS is proposed, designed and partially implemented. WitsCABS employs drivers to perform pick-up and drop-off jobs for customers. In order to request a new job, customers phone in to a central Dispatch Control Centre (DCC) and provide their details, after which, a driver is assigned to their request. A web-application was chosen for the front-end framework with a log-in modal and different views for driver users and DCC users. The driver view allows the driver to view their current jobs and supply feedback to the back-end for when they are on their way and when they have completed an assignment. The DCC view allows dispatch call-centre agents to enter new customers into a form and view currently active jobs. The front-end framework was coded in HTML and angular. The back-end framework was implemented using Apache and SQLite and is a database for storing customer and driver information. 
\end{abstract}

\vfill % Fill the rest of the page with whitespace

\end{titlepage}
\pagenumbering{roman}
\tableofcontents
\newpage
\section{INTRODUCTION}
\pagenumbering{arabic}
\subsection{Project Overview}
A software management system for a taxi company WitsCABS is designed and a basic prototype of the system has been implemented. Sections 2 and 3 address the front-end and back-end frameworks respectively with details for the Software Requirements Specification (SRS) and Software Design Specification (SDS) given for each section. In addition to this, the sections detail how the work was split into sprints according to an Agile, SCRUM framework and how the prototype systems were implemented.
\subsection{Problem Statement}
The WitsCABS company owns a fleet of taxi-cabs and employs drivers for customer transportation within a city. The city is demarcated into service zones with a single taxi-stand in each service zone. Cab drivers wait for clients at their closest taxi-stand. Each driver has a smartphone, embedded with a GPS, maps and navigational tracking and access to the WitsCabs web application. A centralised Dispatch Control Centre (DCC) receives calls from customers and processes the new requests. WitsCABS requires the software application to organise their taxi-fleet, coordinate drivers and manage new job requests.

\subsection{Project Objectives}
\begin{itemize}
\item Create the software requirements specification
\item Create the design documentation for front and back ends
\item Sprint planning  and retrospective for front and back ends
\item Implementation of prototype modules for the front end interfaces
\item Implementation of prototype back-end modules
\item Implementation of an Agile management technique
\end{itemize}

\subsection{Stakeholders}
The project stakeholders are divided into Users, Developers, the Project Manager and the Scrum Master. The respective stakeholders and their roles are documented below.
\subsubsection{Users}
Users are categorised into:
\begin{itemize}
\item Drivers
\item DCC agents
\item Customers
\end{itemize} 
Drivers operate the taxi-cabs, are assigned jobs and pick up and deliver customers. They have access to a driver-specific front-end interface of the WitsCABS application and are able to specify when they accept a job or are on their way as well as state when they have completed a job. Drivers have access to the WitsCABS application through a smart-phone. They also have access to GPS and map navigation and are under navigational tracking. \\\\

DCC agents operate the WitsCABS call centre and have computer access to a DCC-specific interface of the WitsCABS application. They have the ability to add new clients to the database and can view active jobs. \\\\

Customers phone in to the DCC and provide their details, current location and destination. They are assigned a driver, told a likely departure time and given a cost for their journey.

\subsubsection{Developers}
Developers of the WitsCABS application are responsible for the front and back end implementation as well as documentation for the system. The developers are:
\begin{itemize}
\item Frederick Nieuwoudt (front-end)
\item Danielle Winter (front-end)
\item Stephen Friedman (back-end)
\item Sello Molele (back-end)
\end{itemize} 
Frederick Nieuwoudt and Danielle Winter implemented and documented the prototype front end for the system using \textit{angular.js} and \textit{Twitter Bootstrap}. A pair programming approach was used for the implementation.\\\\
Stephen Friedman implemented the back end prototype framework and documentation. Sello Molele contributed to the documentation.
\subsubsection{Project Manager}
The project manager for the WitsCABS project is Danielle Winter. Their responsibilities include organising the development team and managing and integrating the documentation for the project. 
\subsubsection{Scrum Master}
The Scrum Master for the project is Frederick Nieuwoudt. Their responsibilities include managing the sprint backlog and sprint retrospective.   

\newpage
\section{PART A: FRONT-END}
\subsection{SRS}
The software requirements specification quantifies the scope and success criteria for this project. \\ 

The required elements of the front-end of the project as discussed in the SRS that were successfully implemented into the demonstrable prototype include:
\begin{itemize}
\item The log-in page
\item The driver active job view
\item The dispatch agent data entry form
\item The dispatch agent active jobs view
\end{itemize}
With the inclusion of these modules into the prototype allows for the demonstration of the basic work-flow set out for the users of the application.\\

The modules that have yet to be implemented in future sprints that have been laid out in the SRS include:
\begin{itemize}
\item Adding GPS functionality to the application
\end{itemize}

The major downfalls according to the SRS was the lack of test driven development. This was specified in the SRS as the testing plan to be followed.
\subsection{SDS}
The Software Design Specification document details the different components  of the front-end system and gives the relevant modules, controllers, directives and expressions used for the js file. The required components for the HTML code are described. All required data types are stated for mentioned variables. \\\\
The SDS document describes the base functionality of the system, but does not elaborate on any other features to be added at a later date. For example the automatic navigation update in the driver panel has not been elaborated upon. This is due to time constraints in implementing the project and the fact that these additional features were added at a later date. For future work, the SDS must be updated to include these modules.
\subsection{Prototype Implementation}
The basic functionality of the described system was implemented in angular.js and html code. The respective views for the different users of the product were created. Figure 1 and 2 show the DCC view for the system and Figure 3 shows the driver view. The log-in modal view is shown in Figure 4. These views were coded and implemented according to the requirements given in the SDS and SRS. The log-in modal was added as an additional feature for the system. These views satisfactorily represent the front-end product for the WitsCABS application and are a fair indicator of the final system functionality. \\\\

\begin{figure}[!ht]
\centering
\includegraphics[width=1\textwidth]{DCC_active_panel.png}
\caption{DCC view with active customers panel.}
\end{figure}

\begin{figure}[!ht]
\centering
\includegraphics[width=1\textwidth]{new_customer_form.png}
\caption{DCC view with new customer form.}
\end{figure}

\begin{figure}[!ht]
\centering
\includegraphics[width=1\textwidth]{driver_view.png}
\caption{Driver interface view.}
\end{figure}

\begin{figure}[!ht]
\centering
\includegraphics[width=1\textwidth]{login_screen_and_pop_up.png}
\caption{Log-in view with pop-up modal.}
\end{figure}


The front-end is implemented using angular.js and Bootstrap3. Included in the HTML file are URL source links for the angular.js library, the Bootstrap stylesheets and a jQuery plugin. This means that the user merely needs to open the HTML file, which will run if the user has internet access. No additional programs need to be installed. For coding, the front-end framework, the text editor Brackets was used. The front-end views for the driver and DCC interfaces were coded by Danielle Winter and Frederick Nieuwoudt.A pair programming method was used for implementing this system. The work breakdown for the prototype implementation is given below:

\begin{itemize}
\item Frederick Nieuwoudt (Scrum Master):
\begin{itemize}
\item Driver interface implementation
\item Login/log-off
\item Commenting DCC code
\item System aesthetics
\item System modifications
\end{itemize}
\item Danielle Winter (project manager):
\begin{itemize}
\item Driver interface commenting
\item Login/log-off commenting
\item DCC interface implementation
\item System aesthetics
\item System modifications
\end{itemize}
\end{itemize}

\subsection{Product Backlog}
Table 1 shows the product backlog for the front-end of the WitsCABS application including features implemented in the prototype and features still to be added in possible future sprints.\\
\begin{table}[htb]
    \caption{Product Backlog Features.\label{tab:fonts}}
    \begin{center}
\begin{tabular}{|c|l|c|}
\hline 
Interface & Features & Expected Time (Days)\\ 
\hline 
DCC & Active jobs panel & 5\\ 
• & Customer form panel & 5\\ 
• & Implement tabbed view (rather than panels) & 1 \\
• & Automatic address completion using Google Maps API & 5\\

\hline 
Driver & Assigned job panel & 5\\
• & Buttons for "On the way" and "Job complete" & 1\\ 
• & Automatic update of navigation & 5\\
\hline 
General & Log-in page & 5\\ 
• & Log-in authentication & 5\\
• & Create basic interface framework & 1\\
• & Set up AngularJS customer directive & 2\\
• & Review user experience design (aesthetics using CSS) & 5\\
• & Set up google maps API & 5\\
\hline
\end{tabular}
\end{center}
\end{table}


\subsection{Sprint Planning}
Sprint planning was undertaken to divide tasks in such a way that the most efficient path was taken.\\

In order to achieve this the task back log was represented on a gantt chart in order to visualize the product dependencies and which tasks could happen in parallel. The produced gantt chart is available in the figure 5 below.\\

\begin{figure}[ht]
\centering
\label{gantt}
\includegraphics[width=1\textwidth]{gantt.png}
\caption{Gantt chart showing all required tasks and their dependencies.}
\end{figure}

The tasks were divided according to whether they were on the critical path or not. Creating the views for specific users was the most important for the success of the prototype. Tasks such and login authentication and the addition of Google maps functionality were not critical to the creation of a prototype that can visually demonstrate the final product.
\subsection{Sprint Retrospective}
Three sprints were completed to produce the presented front-end prototype. The features implemented in each sprint has been catalogued in Table 2.\\

\begin{table}[htb]
    \caption{Sprint Retrospective for Front-end.\label{tab:fonts}}
    \begin{center} 
\begin{tabular}{|c|l|c|}
\hline 
Sprint & Features\\ 
\hline 
 1 & Create basic interface framework\\ 
• & Set up AngularJS customer directive\\
\hline 
 2 & Driver assigned job panel\\
• & Create default login page\\
• & Create login modal\\
\hline 
 3 & Add DCC active jobs panel \\ 
 • & Add DCC customer form panel\\
\hline
\end{tabular} 
\end{center}
\end{table}



%%%%%%%%%%%%%%%%%%%%%%%%%%%%%%%%%%%%%%%%%%%%%%%%%%%%%%%%%%%%%%%%%%%%%%%%%%%%%%%%%%%%%%%%%%%%%%%%%%%%%%%%%%%%%%%%%%%%%%%%%%%%
\newpage
\section{PART B: BACK-END}
\subsection{SRS}
The Software Requirements Specification (SRS) defines the scope and success criteria of the WitsCABS application. \\

The SRS required the following functionality to be present in the WitsCABS application:
\begin{itemize}
\item DCC Agents must be able to view all active Trips
\item DCC Agents must be able to capture Trips
\item Drivers must be able to view their allocated Trip
\item Drivers must be able to mark Trips as completed
\end{itemize}

The prototype produced for the back-end demonstrates a database capable of storing the information required for the above functionality. It does, however, only contain a script for demonstrating the extracting of information to view all Trips. \\

Further work required is to create scripts capable of extracting or writing any other information required as well as interfacing those scripts with the front-end.\\

The nature of the implementation that was lacking was that it was not test driven. This was specified in the SRS as the testing plan to be followed. Had it been, the script would have been created and the database created to fit the tests.
\subsection{SDS}
The Software Design Specification (SDS) document details the different parts of the back-end that are required, namely the Web Server for interfacing with the front-end and the Database for storing data. \\

Http interfaces are required for the front-end to interact with the Web Server. These are accessed through GET and POST calls from the front-end. The required inputs and the supplied output of each call is defined in the SDS. \\

Once created, databases are difficult to alter without compromising the integrity of the data contained inside its tables, so a strict and robust definition of the database is required. The nature of each table, as well as the structure of each table is defined in the SDS. \\

The SDS lacks clear definition on how third party APIs will be accessed. This is due to time constraints in implementing the project and the fact that these additional features were added at a later date. For future work, the SDS must be updated to include these modules.
\subsection{Prototype Implementation}
The core of the back-end is the database, so it was decided that the demonstration should show the database structure and should contain at least one query. The chosen query was the GetActiveTrips query, as it is the single most complex query required due to the query requiring data from almost all tables.\\

The database structure was created by using database diagrams in Microsoft SQL Management studio, a graphical interface that the developer had previous experience in using. This generated an MSSQL database, which was populated with sample data. A tool called SQLConverter was used to convert the MSSQL database into an SQLite database, from which a database creation script was generated. The database is shown in Figure 6.\\

\begin{figure}[ht]
\centering
\includegraphics[width=1\textwidth]{Database_Diagram.png}
\caption{Database diagram of the prototype.}
\end{figure}

The SQL Query was created in gedit and was designed to illustrate that the data needed to display the DCC Agent view could be retrieved from the database. As the database alone cannot query the Google Maps API, a crude x-y coordinate system was implemented to show that the database is capable of doing calculations. One limitation of SQLite is that is cannot perform square root calculations, so the calculations are left at a point where further maths could be done. \\

\begin{figure}[ht]
\centering
\includegraphics[width=1\textwidth]{Query_Results.png}
\caption{Database prototype query results.}
\end{figure}

In order to run the demonstration, the user must ensure that SQLite is installed. This is done in Ubuntu by running the following in the terminal:
\begin{verbatim}
sudo apt-get sqlite3
\end{verbatim}
In order to import the database and view the demonstration output, the following must be run in the terminal:
\begin{verbatim}
sqlite3
.read CreateSampleDatabase.sql
.read GetActiveJobs.sql
\end{verbatim}

The results of the above query are shown in Figure 7. The demonstration code was created and tested by Stephen Friedman.

\subsection{Product Backlog}
The WitsCABS application Back-End Product Backlog from the point of designing the initial features to the completion of the prototype is shown in Table 3

\begin{table}[htb]
    \caption{Product Backlog of Back-end tasks.\label{tab:fonts}}
    \begin{center} 
\begin{tabular}{|c|l|c|}
\hline 
Component & Task & Expected Time (Days)\\ 
\hline 
Database & Design database structure & 5\\ 
• & Investigate database options & 1\\ 
• & Develop Database Structure & 3 \\
• & Develop Database Queries & 3\\
• & Configure Database & 1\\
• & Build Database Prototype & 5\\
\hline 
Web Server & Design Web Server HTTP interface & 5\\
• & Explore Web Server Options & 1\\ 
• & Develop Web Server & 5\\
• & Configure Server & 2\\
• & Build Server Prototype & 5\\
\hline 
General & Test Integration between Server and Database & 2\\
• & Test Integration between Back-End and Front-End & 2\\
\hline
\end{tabular}
\end{center}
\end{table}

\subsection{Sprint Planning}
Sprint planning was done to ensure that tasks would be divided between the available three sprints. The tasks were placed on Gantt chart in order to figure out which tasks were critical and how tasks could be split into three sprints. The split is shown in Figure 8.

\begin{figure}[!ht]
\centering
\includegraphics[width=1\textwidth]{Back_End_Gantt.png}
\caption{Gantt chart showing the Product Backlog. Critical Path tasks are in red.}
\end{figure}

As both the Web Server and Database tasks are critical, and each task takes 18 days to complete, with two two hour testing tasks. The tasks were split as close to six hours each per sprint as possible.

\subsection{Sprint Retrospective}
Three sprints were completed, producing a back-end prototype. The split of which tasks were accomplished in each sprint are shown in Table 4. 

\begin{table}[!htb]
    \caption{Sprint Retrospective of Tasks for Back-end.\label{tab:fonts}}
    \begin{center} 
\begin{tabular}{|c|l|c|}
\hline 
Sprint & Tasks\\
\hline 
 1 & Design database structure\\ 
• & Investigate database options\\
\hline 
 2 & Develop Database Structure\\
• & Develop Database Queries\\
\hline 
 3 & Configure Database\\ 
 • & Build Database Prototype\\
\hline
\end{tabular} 
\end{center}
\end{table}

\newpage
\section{CONCLUSION}
A software system was designed and implemented using an Agile SCRUM methodology for software management. The designed system prototype demonstrates the expected behaviour for the end product. The WitsCABS application was implemented according to the original SRS and SDS documents and a sprint process was used for realising the product. A concise description of the work breakdown for the project is also given.


\end{document}