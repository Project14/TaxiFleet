%%%%%%%%%%%%%%%%%%%%%%%%%%%%%%%%%%%%%%%%%%%%%%%%%%%%%%%%%%%%%%%%%%%%%%%%%%%%%%%
%
% witseiepaper-2005.tex
%
%                       Ken Nixon (12 October 2005)
%
%                       Sample Paper for ELEN417/455 2005
%
%%%%%%%%%%%%%%%%%%%%%%%%%%%%%%%%%%%%%%%%%%%%%%%%%%%%%%%%%%%%%%%%%%%%%%%%%%%%%%%%

\documentclass[10pt,twocolumn]{witseiepaper}

%
% All KJN's macros and goodies (some shameless borrowing from SPL)
\usepackage{KJN}


%%%%%%%%%%%%%%%%%%%%%%%%%%%%%%%%%%%%%%%%%%%%%%%%%%%%%%%%%%%%%%%%%%%%%%%%%%%%%%%
\begin{document}


\title{FLEET OF TAXI SERVICE MANAGEMENT SYSTEM - LABORATORY EXERCISE 2: SOFTWARE REQUIREMENTS SPECIFICATION}

\author{Danielle Winter (563795), Frederick Nieuwoudt (386372), Stephen Friedman (360938) \& Sello Molele (0604606X)}
\thanks{School of Electrical \& Information Engineering, University of the
Witwatersrand, Private Bag 3, 2050, Johannesburg, South Africa}


%%%%%%%%%%%%%%%%%%%%%%%%%%%%%%%%%%%%%%%%%%%%%%%%%%%%%%%%%%%%%%%%%%%%%%%%%%%%%%%
%

\maketitle
\thispagestyle{empty}\pagestyle{empty}


%%%%%%%%%%%%%%%%%%%%%%%%%%%%%%%%%%%%%%%%%%%%%%%%%%%%%%%%%%%%%%%%%%%%%%%%%%%%%%%
%
\section{Introduction}

\section{User Requirements}
A brief description of the different users and their basic requirements is given. Furthermore, potential users were surveyed for additional requirements. The user requirements are specified based on the MOSCOW and FURPS methodologies \cite{SoftwareEngTextbook}.

\subsection{Basic Customer Requirements}
\begin{itemize}
\item Passengers
\begin{itemize}
\item Registration
\item Account Maintenance
\item Lift Requests
\end{itemize}
\item Drivers
\begin{itemize}
\item Receive Fare
\item Complete Fare
\end{itemize}
\item Administrators
\begin{itemize}
\item Add drivers
\item Maintain driver details
\item View logs
\end{itemize}
\end{itemize}

\subsection{Detailed User Requirements}
Passenger Registration
\begin{itemize}
\item Passenger login details set up
\item Enter personal details (name, age, gender)
\item Enter contact details (email address, cellphone number)
\item Enter banking details (Credit Card number and type, expiry date of card, CVV number)
\end{itemize}
Passenger Account Maintenance
\begin{itemize}
\item Ability to update details (Change in name or contact details and updated payment details)
\item Account cancellation (optional)
\end{itemize}
Passenger Lift Requests
\begin{itemize}
\item Log a destination
\item Log a current location
\item Specify number of passengers to be transported
\item Passenger payment confirmation
\item Order a lift in advance (e.g. collection from the airport at a specified time) 
\item Passenger feedback (ride comfort, satisfaction)
\item Job update (passenger is able to say if they have been collected)
\end{itemize}
Driver Fare Received
\begin{itemize}
\item Driver receives an order for a nearby customer
\item Driver accepts order
\item Passenger updated about taxi arrival time
\item Driver status changed to unavailable
\end{itemize}
Driver  Fare Completed
\begin{itemize}
\item Driver reports drop-off
\item Driver status changed to free
\end{itemize}

Administrator Driver Addition
\begin{itemize}
\item Driver personal details added to database (Name, age, Identity number, gender, medical aid details, tax number)
\item Driver contact details (residential and postal address, email address, cellphone number)
\item Driver vehicle details (number plate, car model, colour and insurance details)
\end{itemize}
Administrator Driver Maintenance
\begin{itemize}
\item Update driver details for personal, contact and vehicle categories
\end{itemize}
Administrator Log Viewing
\begin{itemize}
\item View current job data
\item View historical job data
\end{itemize}

\section{System Specifications}
\subsection{Scope}

The taxi fleet management system is a tool that provides a service where a user can request a taxi and the system will dispatch the closest available driver.\
The system back-end will be required to store information about drivers, customers, and the current pricing scheme.\
The system front end will be presented in the form of a web gui. This will provide a means for customers to input their current location and destination. The web interface will also provide a quoted price and allow for the customer to either accept or reject the quote.

\subsection{Testing Plan}

To achieve a quality product the test driven development(TDD) paradigm will be adopted. TDD calls for the writing of tests before writing the code to be used in the implementation. This encourages simple modular code. The TDD workflow follows the sequence of writing a test, validating that the new test fails, writing the application code, running all the tests together, and then refactoring. This sequence allows for the project to move from getting code working to refactoring the code into simple understandable modules.

\subsection{Implementation Plan}

Following the Scrum SDLC methodology, will provide shippable incrementally improved products. Though a minimum of required features will be needed for a product that can meet the project scope. Additional functionality can be incrementally provided over the life span of the product.




%%%%%%%%%%%%%%%%%%%%%%%%%%%%%%%%%%%%%%%%%%%%%%%%%%%%%%%%%%%%%%%%%%%%%%%%%%%%%%%
%
%\nocite{*}
\bibliographystyle{witseie}
\bibliography{TaxiFleet}

%{\tiny \vfill \hfill \today \hspace{5mm} witseie-paper-2003.\TeX}

\end{document}

" vim: ts=4
" vim: tw=78
" vim: autoindent
" vim: shiftwidth=4
